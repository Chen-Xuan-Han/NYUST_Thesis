
在現代電子製造業中,印刷電路板 (PCB) 的品質檢測是確保產品良率的關鍵環節。傳統的人工目檢方式不僅效率低落,且容易因視覺疲勞導致誤判;而現有的自動光學檢測 (AOI) 設備多基於傳統影像處理演算法,對於複雜背景下的微小瑕疵識別率不佳,且難以適應多樣化的缺陷型態。為解決上述問題,本研究提出了一種基於 YOLOv8 架構的改進型物件偵測演算法,旨在提升 PCB 瑕疵檢測的精確度與即時性。

本研究首先針對 PCB 數據集樣本不平衡的問題,採用 Mosaic 資料增強技術進行預處理;其次,在 YOLOv8 的骨幹網路中引入 CBAM 注意力機制 (Convolutional Block Attention Module),以強化模型對於微小瑕疵特徵的提取能力;最後,利用 EIoU Loss 替換原有的邊界框回歸損失函數,加快收斂速度並提升定位準確度。實驗結果顯示,本方法在公開的 PCB 瑕疵數據集上,平均精確度 (mAP@0.5) 達到了 96.5\%,相較於原始 YOLOv8 模型提升了 3.2\%;同時,檢測速度達到每秒 58 幀 (FPS)。本研究證實了改進後的演算法能有效平衡檢測速度與精度,具備應用於工業流水線即時檢測的潛力。

上文由 GEMINI 生成作為舉例