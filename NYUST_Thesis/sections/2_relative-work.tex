\renewcommand\thetable{\arabic{chapter}-\arabic{table}}
%\renewcommand\thefigure{\arabic{chapter}-\arabic{figure}}
\renewcommand{\theequation}{\arabic{chapter}-\arabic{equation}}
\chapter{Preliminaries}
\label{sec:preliminaries}

近年來,醫療資安事件頻傳,如:2025年某醫美診所遭駭客入侵,顯示出現有的傳統加密與防護機制在面對新型態網路攻擊時,仍存在效率不足或安全漏洞的挑戰。因此,如何建立一個兼具高安全性與傳輸效率的加密/驗證機制,已成為當前急需解決的課題。

在現有的加密標準中,常用的對稱式加密(Symmetric Key Algorithm)AES(Advanced Encryption Standard)雖具有高度安全性並被廣泛採用,但其標準運算流程涉及繁複的矩陣運算與位元組代換,對於運算能力與電池續航力極度受限的物聯網終端節點(如 IoMT 感測器)而言,往往帶來沉重的運算負載與延遲。此外,雖然 RSA(Rivest Shamir Adleman)、ECC (Elliptic Curve Cryptography,橢圓曲線加密)、DSA (Digital Signature Algorithm,數位簽章演算法) 、DH (Diffie-Hellman)等非對稱加密算法能提供身分驗證,但其運算開銷遠高於對稱式加密,難以直接應用於高頻率的即時數據加密。
如圖~\ref{fig:PHI}所示。


\begin{figure}[H] % [t!]塞到該頁的最頂端[H]塞在這裡
  \begin{center}
    \includegraphics[width=1.0\textwidth]{figures/000.png}
    \caption{MY schedule} 
    \label{fig:PHI}
  \end{center}
\end{figure}
