\renewcommand\thetable{\arabic{chapter}-\arabic{table}}
%\renewcommand\thefigure{\arabic{chapter}-\arabic{figure}} 
\chapter{結論與後續工作}
\label{cha:conclusions}


經回顧相關文獻,現有的 AES 優化研究多側重於單一指標的改善,例如單純追求執行速度(Execution Time)的提升,或是專注於記憶體空間(Memory Footprint)的縮減,卻鮮少能同時兼顧能源消耗(Energy Consumption)的表現。考量到 IoMT 裝置多為電池供電,對功耗極為敏感,本研究不僅將針對記憶體與執行效率進行改良,更將「低能耗」納入核心優化目標,致力於在時間、空間與能耗三者之間取得最佳平衡,以符合醫療物聯網之嚴苛需求。

\section{Future Work} 
%\paragraph{Future work} 
本研究預期在 ESP32 嵌入式平台上,成功實作基於混沌映射改良的輕量化 AES 加密系統,並透過實驗數據驗證其在「運算效能」、「能源效率」與「安全性」三方面的綜合優勢。具體預期達成指標如下:


